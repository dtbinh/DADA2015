
\subsection{Management du projet / Project management}
\begin{xcomment}  
Pr\'eciser les aspects organisationnels du projet et les modalit\'es de coordination (si possible individualisation d'une tâche de coordination).
\end{xcomment}


The project will be coordinated by Inria with a collegial decision-making process. A steering committee composed of Thierry Arti\`eres (LIF), 
Georges Gagner\'e (Paris 8), Catherine Pelachaud (LITC) and R\'emi Ronfard (Inria) will make all important decisions  in unanimity. In cases 
of disagreements,  a compromise will have to be found. The committee will meet before every consortium meeting and its decisions will be 
communicated to all consortium members and to ANR. R\'emi Ronfard will be the project coordinator and will be responsible for applying
the decisions of the steering committee and for communicating with ANR and other consortium members. 

Consortium meetings will be organized four times a year for a total of 14 meetings, including kick-off meeting and final review meeting. 
Meetings will take place in the theatre department at Paris 8 to ensure continuity and visibility of the project in the theatre arts world. Paris is also 
a convenient meeting place for all project members. Ensuring regular meetings with all researchers involved in the project is an important
factor for success, given the widely multidisciplinary aspect of the project. This will also favour joint work by the PhD students working
on the project, and ensure their commitment to the user evaluation phase that will take place in Paris 8 as well. 

Software development will be coordinated by Inria and Paris 8 using rapid prototyping methods (SCRUM) \footnote{SCRUM is a popular
and established technique in the game industry, which is applicable to DADA. SCRUM facilitates agile software development
with a small team of part-time developers and a dynamic team organization}. The project coordinator, R\'emi Ronfard,
will act as the "Scrum Master", taking responsibility for the delivery of software increments during fast-paced development periods (including, but not limited to 
the project milestones). C\'edric Plessiet will act as the "Product Owner", taking responsibility for the specifications and selection of features included in each increment,
for the deployment of all software increments at Paris 8, for their testing and user evaluations, and for filling bug reports and user documentations.

All PHD Students will be asked to contribute their latest results to be included in the DADA prototype at least twice a year. 
The rest of their time will be devoted to their research work. Source code files will be signed by all contributing authors, 
together with their affiliations, to properly  track intellectual property rights.



\endinput


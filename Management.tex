
\subsection{Management du projet / Project management}
\begin{xcomment}  
Pr\'eciser les aspects organisationnels du projet et les modalit\'es de coordination (si possible individualisation d'une tâche de coordination).
\end{xcomment}



\subsubsection{WP0. Coordination} 

Four one-day meetings per year -> 14 meetings, including kick-off meeting and final review meeting. 

Collegical decision making:  A steering committee composed of Thierry Arti�res, Georges Gagner�, Catherine Pelachaud  and R�mi Ronfard will make all important decisions  in unanimity. In cases of disagreements, a compromise will have to be found. The committee will meet before every consortium meeting and its decisions will be communicated to all consortium members and to ANR.

Software development will be coordinated by Inria and Paris 8 using rapid prototyping methods.
All PHD Students will be asked to contribute their latest results to be included in the DADA prototype
at least twice a year (two months each). The rest of their time will be devoted to their research work.
References on rapid prototyping would be useful. 

Source code files will be signed by all contributing authors, together with their affiliations, to properly 
track intellectual property rights.

A consortium agreement will be signed in the course of the first year to define a common 
intellectual property rights policy. In order to patent their inventions, partners should seek authorization
from all other partners.  The other partners cannot prevent the patent unless they have prior art. 
The partners  can ask to be mentioned as co-inventors if they can demonstrate that they contriuted to the invention.

The consortium is composed exclusively of academic partners, whose goal is primarily 
to disseminate and publish their research work, not to commercialize software.

On the other hand, some of the inventions may have a commercial value. We will seek advice from
a small board of industry experts  from relevant  French companies (Goalem, Quantum Dream, Ubisoft, Dassault Syst�mes) to detect and address such cases  and make sure that potentially important inventions are protected and made available to them.



\endinput



\section{Description de l'equipe / Team description}
\begin{xcomment}  
A titre indicatif : 2 pages maximum pour ce chapitre.
\end{xcomment}

% 


\subsection{Description, ad\'equation et compl\'ementarit\'e des participants / Partners description, relevance and complementarity}
\begin{xcomment}  
Fournir les \'el\'ements permettant d'appr\'ecier la qualification des personnes impliqu\'ees dans la proposition de projet (le pourquoi qui fait quoi ). Il peut s'agir de r\'ealisations pass\'ees, d'indicateurs (publications, brevets), de l'int\'erêt pour le projet'
Montrer la compl\'ementarit\'e et la valeur ajout\'ee des coop\'erations entre les diff\'erents participants. 
Le cas \'ech\'eant, l'interdisciplinarit\'e et l'ouverture à diverses collaborations seront à justifier en accord avec les orientations du projet.
\end{xcomment}

The consortium involves three research teams with complementary experience in computer graphics, intelligent virtual agents and statistical machine learning and a research team in theatre studies. LTCI and LIF   are already working together on facial animation from speech through the co-supervision of Yu Ding's PHD thesis. Inria/Imagine and Paris 8 are also already working together on directing audiovisual prosody of actors, as part of Ad\'ela Barbulescu's PhD  thesis. Results of the two theses will be exploited in the project.

\subsection{Qualification of the project coordinator}
\begin{xcomment} 
0,5 page maximum
Fournir les \'el\'ements permettant de juger la capacit\'e du coordinateur à coordonner le projet.
\end{xcomment}


{\bf Remi Ronfard} will be the co-ordinator of the DADA project. He is a computer scientist with a 20 year experience in industry and academia in France, Canada and USA. He received his PhD from Mines Paris Tech in 1991 and has worked at the T.J. Watson IBM Research Center in New York as post-doc and visiting scientist (1992 and 2000).   He is the author of 60 scientific papers published in top ranked international journals (IJCV, CVIU, PAMI) and conferences (Siggraph, Eurographics, CVPR, ICCV, ECCV)  and 4 international patents.  He  has conducted research projects in digital storyboarding (INA, 1995), aesthetic surface design (IBM Research and Dassault Syst\'emes, 2000), video indexing (INA, 2000) and statistical analysis of film styles (INRIA 2002-2007). He has been an international expert in the MPEG group (1997-1999) and an R \& D director  at Xtranormal Technologies (2007-2009), where his  team created the patented Magicam system.  He has established active research collaborations with the national  film school (ENS Louis Lumi\'ere) and the Th\'e\^atre des  C\'elestins in Lyon. He  has  co-chaired international workshops on modeling people and human interaction (Beijing, 2005), 3-D cinematography (New York City, 2006; Banff, 2008; Providence, 2012) and intelligent cinematography and editing (Quebec, 2014; Zurich, 2015). He is currently serving as head of the Geometry and Image Department at Laboratoire Jean Kuntzmann, Univ.  Grenoble Alpes.

\subsection{Qualification and contribution of each partner}
\begin{xcomment}  
(1 page maximum)
Qualifier les personnes, pr\'eciser leurs activit\'es principales  et leurs comp\'etences propres.
Pour chacune des personnes dont l'implication dans le projet est sup\'erieure à 25\% de son temps sur la totalit\'e du projet (c'est-à-dire une moyenne de 3 hommes.mois par ann\'ee de projet), une biographie d'une page maximum sera plac\'ee en annexe du pr\'esent document qui comportera :
Nom, pr\'enom, âge, cursus, situation actuelle
Autres exp\'eriences professionnelles
Liste des cinq publications (ou brevets) les plus significatives des cinq derni\`eres ann\'ees, nombre de publications dans les revues internationales ou actes de congr\`es à comit\'e de lecture.
Prix, distinctions
Si besoin, pour chacune des personnes, leur implication dans d'autres projets (Contrats publics et priv\'es effectu\'es ou en cours sur les trois derni\`eres ann\'ees) sera pr\'esent\'ee et fournie en annexe du pr\'esent document. On pr\'ecisera l'implication dans des projets europ\'eens ou dans d'autres types de projets nationaux ou internationaux. Expliciter l'articulation entre les travaux propos\'es et les travaux ant\'erieurs ou d\'ejà en cours.
\end{xcomment}



%\vspace{1cm}
The complete list of permanent researchers involved in the DADA project is summarized in the table below.
%\vspace{1cm}

\begin{tabular}{|p{1.5cm}|p{2cm}|p{1.5cm}|p{2cm}|p{2.5cm}|p{0.5cm}|p{2cm}|}
\hline
Partner &Name & First name & Position & Field of research & PM & Contribution  
\\\hline
Inria & Ronfard & R\'emi &  Researcher & Computer graphics and vision  & 12.6 & WP1, WP2, WP3 (task leader) 
\\\hline
Inria & Cani  &  Marie-Paule &  Professor & Computer Graphics  & 3.6 & WP3 
\\\hline
Inria & Rohmer   &  Damien  &  Assistant Professor & Computer Graphics  & 8.4 & WP3 
\\\hline
Inria & Barbulescu   &  Ad\'ela  &  Post-doc  & Computer Graphics  & 12.6 & WP1, WP2, WP3 
\\\hline
LIF & Arti\`eres& Thierry & Professor & Machine Learning & 14 & WP1 (task leader), WP2, WP3 \\
\hline
LIF & Emyia & Valentin & Assistant Professor & Machine Learning and Signal Processing & 5 & WP1 \\
\hline
LIF & Qi& Wang& Ph.D. student & Machine Learning  & 12 & WP1 
\\\hline
Paris 8 & Gagner\'e & Georges &  Associate Professor & Theatre studies  &  10 & WP3, WP4 (task leader)
\\\hline
Paris 8 & Plessiet & C\'edric & Associate Professor & Computer arts  & 12  & WP3, WP4 
\\\hline
Paris 8 & Dusigne &  Jean-Fran\,cois  & Professor  & Theatre studies  & 3  & WP4 
\\\hline
Paris 8 & L\'egeret &  Katia   &  Professor & Theatre studies  & 1 & WP4 
\\\hline
Paris 8 & Moindrot &   Isabelle &  Professor & Theatre studies  & 1  & WP4 
\\\hline
Paris 8 &  Poirson &  Martial  & Professor  & Theatre studies  & 2  & WP4 
\\\hline
LTCI &  Pelachaud  & Catherine  &  Senior Research   &  Conversational Agents & 12 & WP1, WP2 (task leader), WP3
\\\hline
LTCI &  Clavel & Chlo\'e  &  Assistant Prof. & Conversational Agents   & 3   & WP2 
\\\hline
LTCI &  Ding &  Yu &  Post-doc & Conversational Agents  &  3  & WP1, WP2 
\\\hline

\end{tabular}

\paragraph{INRIA EPI IMAGINE}

IMAGINE is an INRIA research-project-team (EPI) of 6 permanent researchers created in 2012 by Marie-Paule Cani and R\'emi Ronfard.  IMAGINE stands for "Intuitive Modeling and Animation for Interactive Graphics \& Narrative Environments". In adddition to R\'emi Ronfard,  two  permanent  researchers and a post-doctoral student will actively participate to the DADA project.


%These models will be used within different interactive environments dedicated to specific applications. More precisely, we will apply our work to three main domains: shape modeling, motion synthesis and narrative design. In addition to addressing specific needs of digital artists, this research should in the long term, enable professionals and scientists to represent and interact with models of their objects of study, and educators to quickly express and convey their ideas.   Our international scientific partners include UC Berkeley, UBC, the University of  Toronto, McGill and ETHZ and Disney Research Zurich. 


{\bf Marie-Paule Cani} is a Professor of Computer Science at Grenoble University (Grenoble Institute of Technology \& Inria), currently invited professor for 2014-2015 at College de France on the "Informatics and Computational Sciences" chair. Her research interests cover both Shape Modelling and Computer Animation. She contributed over the years to a number of high level models for shapes and motion such as implicit surfaces, multi-resolution physically-based animation and hybrid representations for real-time natural scenes and animated characters - with a specific interest for skin, clothes and hair. Following a long lasting interest for virtual sculpture, she has been recently searching for more efficient ways to create static and animated 3D content such as combining sketch-based interfaces with procedural models expressing a priori knowledge. She received the Eurographics outstanding technical contributions award in 2011 and a silver medal from CNRS in 2012. % for some of this work.

%Marie-Paule Cani served in the program committees of all major conferences in Computer Graphics and was program chair a number of times. She served in the steering committees of SCA, SBIM and SMI, and in the editorial board of Graphical Models, IEEE TVCG and CGF. She is currently associate editor of ACM Transactions on Graphics. She served in the executive committee of ACM SIGGRAPH from 2007 to 2011, and represented Computer Graphics in the ACM Publication Board from 2011 to 2014. In France, she belongs to the executive board of the GDR IG (Informatique Graphique) and to the CA of the French chapter of Eurographics. She has been Vice President of Eurographics since January 2013.

{\bf Damien Rohmer} is assistant professor at CPE Lyon and associate member of the IMAGINE team since 2011 after obtaining his PhD in Computer
Science from Grenoble University. His research interests include  cloth animation and character skinning. He has published papers  in some 
major  Computer Graphics conferences and journals including ACM SIGGRAPH, Eurographics, Symposium on Computer Animation, and Pacific Graphics.  
He will contribute mostly to tasks 3.2 (authoring tools) and 3.3 (real-time character animation). 

{\bf Adela Barbulescu} is a third-year Phd student who will work part-time on the DADA project during her post-doc in 2016.
She will contribute to task 1.2  with her recent work on visual prosody, which will be extended for joint generation of speech
and facial animation from directorial input. 

\paragraph{LIF} Two main researchers and a Ph.D. student from the QARMA team of Computer Science Lab (LIF) at University of Aix-Marseille will participate to the project. 

{\bf Thierry Arti\`eres} is a professor at Ecole Centrale Marseille and member of the {\it QARMA team} (eQuipe AppRentissage et Multim\'edia) at LIF (Laboratoire d'Informatique Fondamentale). One of his major research topic concerns machine learning for multimedia applications, more particularly for sequences and signals, either for classification, pattern discovery, sequence labeling and sequence synthesis, with strong experience with various signals such as speech, bioacoustics, handwriting, gestures, eye movements, WII signals, Kinect and motion capture data. He is author or co-author of about sixty papers and articles in top ranked international conferences (NIPS, ICML, AISTAT, ICASSP, EMNLP) and journals (IEEE PAMI, JMLR, Pattern Recognition) in the fields of theoretical as well as applied machine learning (speech and handwriting recognition, user modeling) and artificial intelligence.  

{\bf Valetin Emiya} is assistant professor in the QARMA team at LIF since 2011. He has conducted research in audio processing and sparse models for 8 years and has strong connexion with the signal processing group at I2M Lab in Marseille. His current works on models and algorithms for audio inpainting. This work is currently being extended to the extrapolation of gesture for the control of electronic musical instrument and contemporary music creation, through the Progest project by GdR ISIS (2014-2016).

% in collaboration with the Centre National de Cr\'eation Musicale \footnote{\url{http://www.gmem.org/index.php option=com_content&view=article&id=5580144&Itemid=13660}}.
 
{\bf Wang Qi} is a first-year Phd student whose research topic on recurrent neural networks for signal processing tasks is related to the project. He will contribute to DADA mainly on WP1 (tasks 1.1 and 1.3). 

\paragraph{Paris 8}

{\bf Georges Gagner\' e} is a lecturer in performing arts at the University Paris 8 where he teaches acting in digital environments. He is also a stage director and member of the collaborative platform \url{didascalie.net}, focusing on real time intermedia environments in performing arts.  In 2007, he initiated the research project ANR VIRAGE about methods and software prototypes for cultural industries and for the arts. He is involved in the OSSIA and INEDIT ANR project through the realization of the artistic project ParOral, based on digital shadows direction through the voice, with the Iscore software. He directed productions in national theaters (TNS, La Filature, Sc\`ene Nationale de Mulhouse, Th\'eatre G\'erard Philipe, Centre dramatique national de Saint-Denis) and organized numerous workshops on the impact of real-time new technologies on theater and scenic writings. He collaborates regularly with St\'ephane Braunschweig and Peter Stein as stage director first assistant  (La Scala, La Fenice, Th\'e\^atre des Champs-Elys\'ees, Op\'era Comique). 

{\bf C\'edric Plessiet} is an associate professor at University Paris 8  where he teaches special fx game programming, Unity 3D programming and computer art. Before that, he has worked with special effects and motion capture companies, creating virtual semi-autonomous butterfly actors for the prize-winning movie "four wings and a pryer". His research interests include movie and theatre previz, real-time game engine programming and artificial intelligence for video games. His computer  art projects have been featured at the Futur En Seine, Paris FX and IVRC conferences. He  has worked with the international Buto dancer Atsuchi Takenuchi. His computer art work was presented in London, Rumania, Switzerland and Japan \footnote{\url{http://www.mobilisimmobilis.com/}, \url{http://ivrc.net/2006/}, \url{http://www.symposium-pi.ch/}}. He will be responsible for deploying, testing, documenting and evaluating the DADA prototypes.

{\bf Jean-Fran�ois Dusigne} is professor of Paris 8 University, and ex-actor of Th��tre du Soleil (Ariane Mnouchkine). He will bring his international expertise on the different ways of directing actors, and the transmission issues.

{\bf Katia L\'egeret}, director of the EA1573 team, "Sc\`enes du monde, cr\'eation et savoirs critiques" 
will help to integrate the resarch goals to the differents themes  of the team, in a transdicisplinary way. 

{\bf Isabelle Moindrot } is the director of the ARTS H2H Labex. She will help to the integration of DADA 
in the global artistic research ecosystem of Paris 8.

{\bf Martial Poirson} is a professor at Paris 8 University, and director of the theatre departement. 
He will help to the dissemination of the DADA deliverables though the academic and professional fields. 



\paragraph{LTCI} LTCI (Laboratoire de Traitement et Communication de l'Information) is a joint laboratory between CNRS and TELECOM ParisTech (UMR 5141). 
%It hosts all the research efforts of TELECOM ParisTech (a faculty of about 150 full-time staff (full professors, associate and assistant professors), 30 full time researchers from CNRS and 300 Ph.D students). 

%Its disciplines include all the sciences and techniques that fall within the term "Information and Communications": Computer Science Networks, Communications, Electronics, Signal and Image Processing, as well as the study of economic and social aspects associated with modern technology. 

{\bf Catherine Pelachaud} is Director of Research at CNRS in the laboratory LTCI, TELECOM ParisTech. She received her PhD in Computer Graphics at the University of Pennsylvania, Philadelphia, USA in 1991. Her research interest includes representation languages for agents, embodied conversational agents, nonverbal communication (face, gaze, and gesture), expressive behaviours and multimodal interfaces. She has been involved and is still involved in several European projects related to multimodal communication (EAGLES, IST-ISLE), to believable embodied conversational agents (IST-MagiCster, FP5 PF-STAR), emotion (FP5 NoE HUMAINE, FP6 IP CALLAS, FP7 STREP SEMAINE) and social behaviours (FP7 NoE SSPNet, H2020 Aria-Valuspa).

{\bf Chlo\'e Clavel} is Assistant Professor at Telecom Paristech. She owned a PhD on acoustic analysis of emotional speech. Before joining Telecom ParisTech she worked as a researcher at Thales Research and Technology where she focused on emotion analysis; then she became a researcher at EDF R \& D working on sentiment analysis and opinion mining. She has participated to several collaborative projects and has coordinated one national project.

{\bf Yu Ding} is a post-doctoral student at LTCI. He has obtained his PhD in September 2014 under the supervision of Thierry Arti�res and Catherine Pelachaud. His topics of interest are to develop data-driven approach for expressive animation of virtual agents.



%References
%
%Magalie Ochs, Catherine Pelachaud, {\bf Socially Aware Virtual Characters: The Social Signal of Smiles}. IEEE Signal Processing Magazine 30(2): 128-132 (2013)
%
%Nesrine Fourati, Catherine Pelachaud, Multi-level classification of emotional body expression, IEEE International Conference on Automatic Face and Gesture Recognition FG'15, Ljubljana, Slovenia, May 2015.
%
%Brian Ravenet, Magalie Ochs, Catherine Pelachaud, Architecture of a Socio-Conversational Agent in Virtual Worlds, IEEE International Conference on Image Processing (ICIP2014), Paris, France, October 2014.
%
%Nesrine Fourati, Catherine Pelachaud: Emilya: Emotional body expression in daily actions database. Language Resources and Evaluatin Conference LREC, Reykjavic, Iceland, May 2014: 3486-3493




\endinput


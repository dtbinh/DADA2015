
\section{Description de l'equipe / Team description}
\begin{xcomment}  
A titre indicatif : 2 pages maximum pour ce chapitre.
\end{xcomment}


\subsection{Description, ad\'equation et compl\'ementarit\'e des participants / Partners description, relevance and complementarity}
\begin{xcomment}  
Fournir les \'el\'ements permettant d'appr\'ecier la qualification des personnes impliqu\'ees dans la proposition de projet (le pourquoi qui fait quoi ). Il peut s'agir de r\'ealisations pass\'ees, d'indicateurs (publications, brevets), de l'int\'erêt pour le projet'
Montrer la compl\'ementarit\'e et la valeur ajout\'ee des coop\'erations entre les diff\'erents participants. 
Le cas \'ech\'eant, l'interdisciplinarit\'e et l'ouverture à diverses collaborations seront à justifier en accord avec les orientations du projet.
\end{xcomment}

The consortium involves three research teams with complementary experience in computer graphics, intelligent virtual agents and statistical machine learning and a research team in theatre studies. Telecom ParisTech and University of Marseille are already working together on facial animation from speech through the co-supervision of Yu Ding's thesis (Ding 2013). Inria/Imagine and Paris 8 are also already working together on directing audiovisual prosody of actors, as part of Ad\'ela Barbulescu thesis (Barbulescu 2014). Results of the two theses will be exploited in the project.

\subsection{Qualification du coordinateur du projet / Qualification of the project coordinator}
\begin{xcomment} 
0,5 page maximum
Fournir les \'el\'ements permettant de juger la capacit\'e du coordinateur à coordonner le projet.
\end{xcomment}

R\'emi Ronfard is a senior researcher at Inria in the IMAGINE team, whose research is devoted to designing novel interfaces between artists and computers (Intuitive Modeling and Animation for Interactive Graphics \& Narrative Envi-ronments). He has a 20 year experience in industry and academia in France, Canada and USA, and has directed an R \& D team on virtual cinematography at Montreal-based startup Xtranormal Technologies.  He will be acting as coordinator of DADA.



\subsection{Qualification, rôle et implication des participants / Qualification and contribution of each partner}
\begin{xcomment}  
(1 page maximum)
Qualifier les personnes, pr\'eciser leurs activit\'es principales  et leurs comp\'etences propres.

Pour chacune des personnes dont l'implication dans le projet est sup\'erieure à 25\% de son temps sur la totalit\'e du projet (c'est-à-dire une moyenne de 3 hommes.mois par ann\'ee de projet), une biographie d'une page maximum sera plac\'ee en annexe du pr\'esent document qui comportera :
Nom, pr\'enom, âge, cursus, situation actuelle
Autres exp\'eriences professionnelles
Liste des cinq publications (ou brevets) les plus significatives des cinq derni\`eres ann\'ees, nombre de publications dans les revues internationales ou actes de congr\`es à comit\'e de lecture.
Prix, distinctions
Si besoin, pour chacune des personnes, leur implication dans d'autres projets (Contrats publics et priv\'es effectu\'es ou en cours sur les trois derni\`eres ann\'ees) sera pr\'esent\'ee et fournie en annexe du pr\'esent document. On pr\'ecisera l'implication dans des projets europ\'eens ou dans d'autres types de projets nationaux ou internationaux. Expliciter l'articulation entre les travaux propos\'es et les travaux ant\'erieurs ou d\'ejà en cours.

\end{xcomment}
\begin{table}
\begin{tabularx}{ \textwidth}{| p{2cm} | p{2cm} |  p{2cm} |  p{2cm} |  p{1cm} |  X |    }
\hline
Name & First name & Position & Field of research & PM & Contribution to the proposal \\
\hline
\end{tabularx}

\caption{Qualification and contribution of each partner}
\end{table}

Thierry Arti\`eres  is a professor at University of Aix-Marseille, and a member of the QARMA team (eQuipe Ap-pRentissage et Multim\'edia) at LIF (Laboratoire d'Informatique Fondamentale). One of his major research topic concerns machine learning for multimedia applications, more particularly for sequences and signals, either for classification, pattern discovery, sequence labeling and sequence synthesis, with strong experience with various signals such as speech, bioacoustics, handwriting, gestures, eye movements, WII signals, Kinect and motion capture data. 

Georges Gagner\'e is stage director (www.didascalie.net) and associate professor in Paris 8 University's performing arts department, working in the laboratory "Sc\`enes du monde, cr\'eation, savoirs critiques" (EA 1573), with full professor Jean-François Dusigne, ex-actor of Th\'e\^atre du Soleil, and international expert in directing actor theory and practice. He works closely with the digital artist and associated professor C\'edric Plessiet from Paris 8 's INREV research laboratory (EA4010- digital image and virtual reality), directed by Marie-H\'el\`ene Tramus, full professor and scientific director of Labex Arts and Human Mediations (www.labex-arts-h2h.fr) linking together artistic practice with cognitive sciences and human mediations.

Catherine P\'elachaud is Director of Research at CNRS in the laboratory LTCI, TELECOM ParisTech. She has pub-lished over 150 papers and chapters in internationally recognized conferences and journals. She has participated in several national and European projects related to multimodal communication, to believable embodied conversational agents, emotions and social behaviors. She has developed an open-source virtual agent system, Greta, which is used by several international teams for research and teaching purposes. 

\endinput


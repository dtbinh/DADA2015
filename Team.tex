
\section{Description de l'equipe / Team description}
\begin{xcomment}  
A titre indicatif : 2 pages maximum pour ce chapitre.
\end{xcomment}

% 


\subsection{Description, ad\'equation et compl\'ementarit\'e des participants / Partners description, relevance and complementarity}
\begin{xcomment}  
Fournir les \'el\'ements permettant d'appr\'ecier la qualification des personnes impliqu\'ees dans la proposition de projet (le pourquoi qui fait quoi ). Il peut s'agir de r\'ealisations pass\'ees, d'indicateurs (publications, brevets), de l'int\'erêt pour le projet'
Montrer la compl\'ementarit\'e et la valeur ajout\'ee des coop\'erations entre les diff\'erents participants. 
Le cas \'ech\'eant, l'interdisciplinarit\'e et l'ouverture à diverses collaborations seront à justifier en accord avec les orientations du projet.
\end{xcomment}

The consortium involves three research teams with complementary experience in computer graphics, intelligent virtual agents and statistical machine learning and a research team in theatre studies. Telecom ParisTech and University of Marseille are already working together on facial animation from speech through the co-supervision of Yu Ding's thesis (Ding 2013). Inria/Imagine and Paris 8 are also already working together on directing audiovisual prosody of actors, as part of Ad\'ela Barbulescu thesis (Barbulescu 2014). Results of the two theses will be exploited in the project.






\subsection{Qualification du coordinateur du projet / Qualification of the project coordinator}
\begin{xcomment} 
0,5 page maximum
Fournir les \'el\'ements permettant de juger la capacit\'e du coordinateur à coordonner le projet.
\end{xcomment}


Remi Ronfard is a computer scientist with a 20 year experience in industry and academia in France, Canada and USA. He has worked at the T.J. Watson IBM Research Center in New York as post-doc and as a visiting scientist (1992 and 2000).  He is now a member of the IMAGINE research team at Inria and the University of Grenoble, where his research is devoted to designing novel interfaces between artists and  computers. He is the author of 4 international patents and more than 60 scientific papers published in top ranked international journals (IJCV, CVIU, PAMI) and conferences (Siggraph, Eurographics, CVPR, ICCV, ECCV)  and cited more than 4000 times.  He will be acting as coordinator for DADA.

Remi was trained as an engineer then PhD student at Mines Paris Tech. He  has conducted research in a variety of domains, including digital storyboarding (INA, 1995), aesthetic surface design (IBM Research and Dassault Syst�mes, 2000), video indexing (INA, 2000), action recognition and statistical analysis of image and film styles (INRIA 2002-2007). He was an expert in the international MPEG group from 1997 to 2000. In 2007, he became a team leader at Xtranormal Technologies. His  team created the patented ?magicam? system, which was used to produce two million user-generated 3D animation movies. He came back to INRIA in 2009 with a new research program devoted to  ?directing virtual worlds?. He helped to create the IMAGINE team in 2012,  where he now  leads the ?narrative design? part of the project. Towards this goal, he investigates computational models of visual storytelling. This has led to inspiring collaborations with the national  film school (ENS Louis Lumi�re) and the C�lestins Theatre in Lyon. He 
has  co-chaired international workshops on modeling people and human interaction (Beijing, 2005), 3-D cinematography (New York City, 2006; Banff, 2008; Providence, 2012), intelligent cinematography and editing (Quebec, 2014). He is currently serving as head of the ?Geometry and Image? Department at Laboratoire Jean Kuntzmann, Univ.  Grenoble Alpes.



\subsection{Qualification, rôle et implication des participants / Qualification and contribution of each partner}
\begin{xcomment}  
(1 page maximum)
Qualifier les personnes, pr\'eciser leurs activit\'es principales  et leurs comp\'etences propres.

Pour chacune des personnes dont l'implication dans le projet est sup\'erieure à 25\% de son temps sur la totalit\'e du projet (c'est-à-dire une moyenne de 3 hommes.mois par ann\'ee de projet), une biographie d'une page maximum sera plac\'ee en annexe du pr\'esent document qui comportera :
Nom, pr\'enom, âge, cursus, situation actuelle
Autres exp\'eriences professionnelles
Liste des cinq publications (ou brevets) les plus significatives des cinq derni\`eres ann\'ees, nombre de publications dans les revues internationales ou actes de congr\`es à comit\'e de lecture.
Prix, distinctions
Si besoin, pour chacune des personnes, leur implication dans d'autres projets (Contrats publics et priv\'es effectu\'es ou en cours sur les trois derni\`eres ann\'ees) sera pr\'esent\'ee et fournie en annexe du pr\'esent document. On pr\'ecisera l'implication dans des projets europ\'eens ou dans d'autres types de projets nationaux ou internationaux. Expliciter l'articulation entre les travaux propos\'es et les travaux ant\'erieurs ou d\'ejà en cours.

\end{xcomment}
\begin{table}
\begin{tabularx}{ \textwidth}{| p{2cm} | p{2cm} |p{2cm} |  p{2cm} |  p{2cm} |  p{1cm} |  X |    }
\hline
Partner &Name & First name & Position & Field of research & PM & Contribution  \\
\hline
LIF & Arti\`eres& Thierry & Pr & Machine Learning & 14 & WP1 (task leader), WP2, WP3 \\
\hline
LIF & Emyia & Valentin & Assistant Pr& Machine Learning and Signal Processing & 5 & WP1 \\
\hline
LIF & Qi& Wang& Ph.D. student & Machine Learning  & 12 & WP1 \\
\hline

\end{tabularx}

\caption{Qualification and contribution of each partner}
\end{table}

\paragraph{INRIA EPI IMAGINE}

IMAGINE stand for: "Intuitive Modeling and Animation for Interactive Graphics \& Narrative Environments". The challenge we aim to address is the efficient, interactive creation of animated 3D content. To this end, our goal is to develop a new generation of knowledge-based models for shapes, motions and stories. These models will embed both procedural methods, enabling the fast generation of high quality content, and intuitive control handles, enabling users to easily convey their intent and to progressively refine their result. These models will be used within different interactive environments dedicated to specific applications. More precisely, we will apply our work to three main domains: shape modeling, motion synthesis and narrative design. In addition to addressing specific needs of digital artists, this research should in the long term, enable professionals and scientists to represent and interact with models of their objects of study, and educators to quickly express and convey their ideas.  
Our international scientific partners include UC Berkeley, UBC, the University of  Toronto, McGill and ETHZ and Disney Research Zurich. 

In adddition to Remi Ronfard, two other permanent  researchers and a post-doctoral student will actively participate to the DADA project.

{\bf Marie-Paule Cani} is a full-time professor at INPG and director of the IMAGINE team. She will contribute to DADA 
with her recent work on implicit skinning, advanced hair style rendering, advanced clothe adaptation, etc. 

{\bf Damien Rohmer} is an associate member of the IMAGINE team. He will contribute to DADA with his recent work on
implicit skinning ? physically-based animation ? 

{\bf Adela Barbulescu} is a third-year Phd student who will work part-time on the DADA project during her post-doc in 2016.
She will contribute to DADA with her recent work on visual prosody, which will be extended for joint generation of speech
and facial animation from directorial input. 


\paragraph{ECM} Two main researchers from the QARMA team will participate to the project. 

{\bf Thierry Arti\`eres} is a professor at University of Aix-Marseille, and a member of the {\it QARMA team} (eQuipe AppRentissage et Multimédia) at LIF (Laboratoire d’Informatique Fondamentale). One of his major research topic concerns machine learning for multimedia applications, more particularly for sequences and signals, either for classification, pattern discovery, sequence labeling and sequence synthesis, with strong experience with various signals such as speech, bioacoustics, handwriting, gestures, eye movements, WII signals, Kinect and motion capture data. He is author or co-author of about sixty papers and articles in top ranked international conferences (NIPS, ICML, AISTAT, ICASSP, EMNLP) and journals (IEEE PAMI, JMLR, Pattern Recognition) in the fields of theoretical as well as applied machine learning (speech and handwriting recognition, user modeling) and artificial intelligence.  

{\bf Valetin Emiya} is assistant professor in the QARMA team at LIF since 2011. He has conducted research in audio processing and sparse models for 8 years and has strong connexion with the signal processing group at I2M Lab in Marseille. His current works on models and algorithms for audio inpainting (see \cite{Adler2012, Adler2011} and project ANR JCJC MAD), i.e. interpolation and extrapolation in audio sequences. This works are currently being extended to the extrapolation of gesture for the control of electronic musical instrument and contemporary music creation, through the Progest project by GdR ISIS (2014-2016) in collaboration with the gmem Centre National de Cr\'eation Musicale (\url{http://www.gmem.org/index.php?option=com_content&view=article&id=5580144&Itemid=13660}).
 
 
{\bf Wang Qi} is a first-year Phd student whose research topic on recurrent neural networks for signal processing tasks is related to the project. He will contribute to DADA mainly on WP1 (tasks 1.1 and 1.3). 

\paragraph{Paris 8}

{\bf Georges Gagner�} is a lecturer in performing arts at the University Paris 8 where he teaches acting in digital environments. As part of Labex Arts H2H, he is an active member of two projects�on  ��Actor directing as art creation process�� (La direction d?acteurs comme processus de cr�ation artistique), and ��Augmented scenery�� (La Sc�ne Augment�e). He is also a stage director and member of the collaborative platform didascalie.net, focusing on real time intermedia environments in performing arts.  In 2007, he initiated the research project ANR VIRAGE about methods and software prototypes for cultural industries and for the arts. He is involved in the OSSIA and INEDIT ANR project through the realization of the artistic project ParOral, based on digital shadows direction through the voice, with the Iscore software. He directed productions in national theaters (Th��tre National de Strasbourg, La Filature, Sc�ne nationale de Mulhouse, Th��tre G�rard Philipe, Centre dramatique national de Saint-Denis) and organized numerous workshops on the impact of real time new technologies on theater and scenic writings. He collaborates with St�phane Braunschweig and Peter Stein as stage director first assistant on more than 20 differents opera productions in the most famous european theaters (La Scala, La Fenice, Th��tre des Champs-Elys�es, L'Op�ra Comique, Le Festival International d'Art Lyrique d'Aix-en-Provence, La Monnaie de Bruxelles, L'Op�ra de Lyon, L'Op�ra du Rhin, etc.). 

{\bf C�dric Plessiet} is an associate professor at University Paris 8  where he teaches special fx game programming, Unity 3D programming and computer art. Before that, he has worked with special effects and motion capture companies, creatin virtual semi-autonomous butterfly actor for prize-winning movie "four wings and a pryer". His research interests include movie and theatre previz, real-time game engine programming and artificial intelligence for video games. His computer  art projects have been featured at the Futur En Seine, Paris FX and IVRC conferences. He  has worked with the international Buto dancer Atsuchi Takenuchi. His art was presented in London, Rumania Switzerland and Japan (http://www.mobilisimmobilis.com/, http://ivrc.net/2006/, http://www.symposium-pi.ch/)

{\bf Jean-Fran�ois Dusigne} is  professor of Paris 8 University, and ex-actor of Th��tre du Soleil (Ariane Mnouchkine). He will bring his international expertise on the different ways of directing actors, and the transmission issues.

{\bf Isabelle Moindrot} is the  director of the ARTS H2H Labex. She  will help to the integration of DADA in the global artistical research ecosystem of Paris 8.

{\bf Martial Poirson} is a professor at Paris 8 University, and director of the theatre departement. He will help to the dissemination of the DADA's deliverables though the academic and professional fields.


\paragraph{LTCI} LTCI (Laboratoire de Traitement et Communication de l?Information) is a joint laboratory between CNRS and TELECOM ParisTech (UMR 5141). It hosts all the research efforts of TELECOM ParisTech (a faculty of about 150 full-time staff (full professors, associate and assistant professors), 30 full time researchers from CNRS and 300 Ph.D students). Its disciplines include all the sciences and techniques that fall within the term "Information and Communications": Computer Science Networks, Communications, Electronics, Signal and Image Processing, as well as the study of economic and social aspects associated with modern technology. 

{\bf Catherine Pelachaud} is Director of Research at CNRS in the laboratory LTCI, TELECOM ParisTech. She received her PhD in Computer Graphics at the University of Pennsylvania, Philadelphia, USA in 1991. Her research interest includes representation languages for agents, embodied conversational agents, nonverbal communication (face, gaze, and gesture), expressive behaviours and multimodal interfaces. She has been involved and is still involved in several European projects related to multimodal communication (EAGLES, IST-ISLE), to believable embodied conversational agents (IST-MagiCster, FP5 PF-STAR), emotion (FP5 NoE HUMAINE, FP6 IP CALLAS, FP7 STREP SEMAINE) and social behaviours (FP7 NoE SSPNet, H2020 Aria-Valuspa).

{\bf Chlo� Clavel} is Assistant Professor at Telecom Paristech. She owned a PhD on acoustic analysis of emotional speech. Before joining Telecom ParisTech she worked as a researcher at Thales Research and Technology where she focused on emotion analysis; then she became a researcher at EDF R \& D working on sentiment analysis and opinion mining. She has participated to several collaborative projects and has coordinated one national project.

{\bf Yu Ding} is a post-doctoral student at LTCI. He has obtained his PhD in September 2014 under the supervision of Thierry Arti�res and Catherine Pelachaud. His topics of interest are to develop data-driven approach for expressive animation of virtual agents.


\endinput


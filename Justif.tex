
\section{Justification scientifique des moyens demand\'es / Scientific justification of requested ressources}
\begin{xcomment}  
On pr\'esentera ici la justification scientifique et technique des moyens demand\'es dans le document de soumission tel que synth\'etis\'e et rempli en ligne sur le site de soumission dans la fiche tableaux r\'ecapitulatifs  du document administratif et financier tel que rempli en ligne sur le site de soumission.
Justifier les moyens demand\'es en distinguant les diff\'erents postes de d\'epenses.
(2 pages maximum)
\end{xcomment}


Budget: We request a financial aid of 518  K\euro  for three PhD students  and two post-doctoral students,  computer hardware and software, and
travel expenses (50 K\euro{}). The project duration should be 42 months in order to develop a functional prototype in three years 
and to use it to animate several play scripts for evaluation and validation purposes.









\subsection{\'Equipement / Equipment}
\begin{xcomment}  
Pr\'eciser la nature des \'equipements* et justifier le choix des \'equipements (un devis pourra être demand\'e si le projet est retenu pour financement).
Dans le cas où les achats doivent être compl\'et\'es par d'autres sources de financement, indiquer le montant et l'origine de ces aides compl\'ementaires, et le pourcentage demand\'e à l'ANR sur le pr\'esent projet.
\end{xcomment}


LTCI request 6 500 \euro{}  to  cover equipments (High performance PC with good graphics card).  LIF and Inria each request 5 K\euro{}  to  cover equipments (work station with high quality video card enabling Graphical Processing Unit computation) and necessary softwares,  and research books. The theater laboratory EA1573 asks for 4000 \euro{} for the purchase of a working station and all the softwares and hardwares necessary for the development of the WP4.

\subsection{Personnel / Staff}
\begin{xcomment}  
Le personnel non permanent (th\`eses, post- doctorants, CDD...) financ\'e sur le projet devra être justifi\'e.
Fournir  les profils des postes à pourvoir pour les personnels à recruter.
Pour les th\`eses, pr\'eciser si des demandes de bourse de th\`ese sont pr\'evues ou en cours, en pr\'eciser la nature et la part de financement imputable au projet. 
\end{xcomment}

LTCI requests a financial aid of 108 K\euro{}  for a PhD. The PhD topic  will focus on modeling group behaviors of virtual agents while conversing or performing actions together (eg following each other). The Phd Student  will also develop algorithm to merge procedural and statistical animations.

LIF requests a financial aid of 101 K\euro{}  for a PhD. His work will be focused on WP1 and more particularly on statistical machine learning for designing generic controlers (task 1.1 and 1.3). LIF requests an additional 6K \euro for hiring two 5 months internships in order to perform more exploratory studies.

Inria requests a financial aid of 117 K\euro{}  for a PhD. The PhD Topic will focus on  a real-time synchronization model for virtual actors performing
coordinated movements. The thesis will be devoted to designing and implementing the dramatic score language central to WP3 and applying it
to create believable multi-character 3D animation in the Unity game engine.
  
The theater research team EA1573 asks for two Post-Doc of 12 months each (40k \euro{} per Post-Doc). In collaboration with the INREV team, the first post-doc 
will intervene on the WP4.2 to guarantee an effective communication between the WP3 and the WP4  but also to help in the software implementation of experiments, documentation by tutorial, sample and user's manual. The second post-doc will intervene on the WP4.3 to bring to a successful conclusion of the dramatic  score, to build complex animation, and to help in the validation of the prototype whether it is from a qualitative and quantitative point of view with the spectators, the actors and the users of the software. It will also have for mission to finalize the user set of documents to guarantee the distribution of the system to other potential users (video games, performing arts, etc.)

%\subsection{Prestation de service externe / Subcontracting}
%\begin{xcomment}  
%Pr\'eciser:
%la nature des prestations,
%le type de prestataire.
%\end{xcomment}

\subsection{Missions / Travel}
\begin{xcomment} 
Pr\'eciser :
les missions li\'ees aux travaux d'acquisition sur le terrain (campagnes de mesures'),
les missions relevant de colloques, congr\`es'
\end{xcomment}

LTCI, Inria and LIF each request a budget of 15K \euro{}  to participate in project meetings, exchange PhD students over short periods of time,
and participate in national and international conferences. Paris 8 requests  4000 \euro{}  in spawn of mission to allow the researchers mobilized to exchange with the partner labs, but also to present their research in various colloquiums and festivals (Ars Electronica, Laval Virtual, Les bains num�riques, SIGGRAPH Art Sessions). Theater laboratory asks for 4000 \euro{} to finance external expertises of the theater field to help in the validation of the deliverables.

%\subsection{D\'epenses justifi\'ees sur une proc\'edure de facturation interne / Costs justified by internal procedures of invoicing}
%\begin{xcomment}  
%Pr\'eciser la nature des prestations.
%\end{xcomment}
%Not applicable.
%
\subsection{Autres d\'epenses de fonctionnement / Other expenses}
\begin{xcomment}
Toute d\'epense significative relevant de ce poste devra être justifi\'ee.
\end{xcomment}
A small additional budget is requested by all partners to cover publication fees for open-access journals,  software and research books 
supporting the work of PhD students, as well as internships on topics relevant to DADA during the three academic years covered by the project. 




\endinput


\subsubsection{WP2. Proxemic component: procedural animation models for interaction between actors.}

Previous works on modeling group formation have been mainly applied to ECAs and have focused on the spatial positioning and orientation of the ECAs \cite{Pedica2010}. Few researches have looked at modeling group of ECAs with dif-ferent personalities and social attitudes \cite{Gillies2004,Prada2005}. However these models do not consider the dynamic evolution of the group behaviors nor how do the actors' behaviours synchronize with each other. In this task, we focus on simulating group of autonomous actors interacting with each other where each actor is defined by its emotional state and its relation toward others and objects. Social relations can be represented by two dimensions, affiliation and dominance \cite{Wiggins1979}. We will extend group behavior model \cite{Pedica2010} that embeds the F-Formation proposed by Kendon \cite{Kendon2004} to consider social relations and emotional states of actors. 

Physical distance between actors, their body orientation toward each other, gaze direction, facial expression, gesture expressivity are cues of the relation with others and with objects and of emotional states. These cues will be embedded in the proxemics component. They evolve continuously in relation to the others' behaviors. To simulate the dynamic evolution of these behaviors we will make use of Neural Network simulation \cite{Prepin2013} where we can render how behaviors of one actor can act on behaviors of other actors (eg walking powerfully toward an actor with an angry expression will result in moving backward of another actor with a less dominant attitude. Mutual coupling of behaviors will be modeled as emerging from such action-reactive behavior simulation \cite{Prepin2013} ensuring not only the synchronization between actors' behaviors but also their mutual influence. This task will be led by Telecom ParisTech with the contribution of Inria.



\endinput
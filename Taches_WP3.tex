\subsubsection{WP3 Authoring}


\begin{center}
\begin{tabular}{|l|l|}\hline
WP3 &  Authoring \\\hline
Responsable &  Inria  \\\hline
Participants &  Paris 8, ECM, Telecom ParisTech\\\hline
Duration  &   \\\hline
Objectives &   \\\hline
Content &  \\\hline
Task 31 & Blocking language  \\\hline
Task12 &  Authoring tools \\\hline
Task13 &  Real-time animation  \\\hline
\end{tabular}
\end{center}


\paragraph{Task31: Specification of a dramatic language for virtual actors.}

This will include a choice of verbs (actions, speech acts, movements) and adverbs (moods, attitudes, dramatic effects) 
for directing actors ; define cues as synchronisation points between actors ; define parallel and sequential behaviors ; etc.

Part of this language will be devoted to stage blocking / movement

Part of this language will be devoted to dialogue

Previous work \cite{gagnere2012,ronfard2012,gagnere2015}.

In theatre, blocking is the precise movement and staging of actors on a stage in order to facilitate the performance of a play, ballet, film or opera.

A theatrical cue is the trigger for an action to be carried out at a specific time. It is generally associated with theatre and the film industry. They can be necessary for a lighting change or effect, a sound effect, or some sort of stage or set movement/change.

A cue sheet is a form usually generated by the stage manager or design department head that indicates information about the cue including execution, timing, sequence, intensity (for lights), and volume (for sound). The stage manager keeps a master list of all the cues in the show and keeps track of them in the prompt book.


The prompt book, also called prompt book, transcript, the bible or sometimes simply "the book," is the copy of a production script that contains the information necessary to create a theatrical production from the ground up. It is a compilation of all blocking, business, light, speech and sound cues, lists of properties, drawings of the set, contact information for the cast and crew, and any other relevant information that might be necessary to help the production run smoothly and nicely.


The Prompt Book is the master copy of the script or score, containing all the actor moves and technical cues, 
and is used by the deputy stage manager to run rehearsals and later, control the performance.



\paragraph{Task32: Authoring tools for blocking a scene with multiple actors.}

Design and implementation of authoring tools for creating animation with the dramatic language.

Previous work has focused on direct annotation of play-scripts with high-level (FML) or low-level (BML) mark-up.

From a user perspective, this is neither intuitive nor expressive. Instead, we will offer authoring tools
with natural interaction, taking inspiration from existing practices in theatre (prompt-books, cue sheet, storyboards, etc.).


The authoring tool may include multimodal interaction with the director: sketching tools for designing actor trajectories
and meeting points; writing tools for adding didascalia to dialogues; timeline-driven interaction for defining cue points
and actions, timing, etc.

User interface for directing actors by sketching stage floor plans and composing the dramatic score ; one line per actor per motion component (proxemic behaviors, kinesic actions, kinesic moods, speech acts, etc.)

Compilation of the language into a finite state machine and/or Petri net ; allowing real-time execution of the dramatic score.



\paragraph{Task33: Real-time execution of the dramatic score.} 

This should include real-time combination of proxemic (procedural) and kinesic components of motion ; non-deterministic motion generation ; synchronization to cues ; real-time skinning and advanced 3D animation ; integration of physically-based secondary animation (skin, hair, clothes, etc.)

This includes integration of the GRETA BML realizer with IMAGINE animation ; and real-time integration of the statistical models of motion with the procedural animation components.


One challenge to be overcome is in combining full 	body animation and interaction animation at runtime.

We believe it will be an importa asset for the DADA platform that each performance is unique,
and can be controled in real time by cues given by the director. 
  
We will pay particular attention to design models capable of generating real animations. Indeed synthezing from statistical models usually resumes to finding the most likely animation sequence in a given situation, which may yield to too similar and unrealistic animations. 

Actually one would be pretty much interested in synthezing animations that are both likely given the learnt statistical models but also exhibiting the variability one can observe in human motion and gestures. Introducing such a stochastic component in the synthesis while maintaining a high quality animation level is not 
straightforward and is an open question that we will have to solve.

\paragraph{Partners' roles}

Inria will be the main software developper.

Task 3.1 will be jointly performed by Inria, LITC and Paris 8.

LIF will contribute to task 3.3 on implementing non-deterministic animation 
methods using statistical models trained in WP1.

Paris 8 will contribute to tasks 3.2 and 3.3 by being the "product owner" for the authoring tool.

LITC will contribute to task   3.3 by providing a subset of the GRETA platform.





\paragraph{Deliverables}

bla bla

\begin{tabular}{|l|l|l|}\hline
Deliverables & Name and content  & Date  \\\hline
L1.1  & Report on the state of the art for virtual theatre & \\\hline
L1.2  &  & \\\hline
L1.3  &  & \\\hline
\end{tabular}


\endinput

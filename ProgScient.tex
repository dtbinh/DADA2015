
\section{Programme scientifique et technique, organisation du projet / Scientific and technical programme, Project organisation}
\begin{xcomment}  
A titre indicatif : de 5 à 10  pages pour ce chapitre, en fonction du nombre de tâches
\end{xcomment}

\subsection{Programme scientifique et structuration du projet  / Scientific programme, project structure}
\begin{xcomment}  
 Pr\'esentez le programme scientifique et justifiez la d\'ecomposition en tâches du programme de travail en coh\'erence avec les objectifs poursuivis. 
Utilisez un diagramme pour pr\'esenter les liens entre les diff\'erentes tâches (organigramme technique)
Les tâches repr\'esentent les grandes phases du projet. Elles sont en nombre limit\'e.
Le cas \'ech\'eant (programmes exigeant la pluridisciplinarit\'e), d\'emontrer l'articulation entre les disciplines scientifiques.
N'oubliez pas les tâches correspondant à la diss\'emination et à la valorisation, à d\'ecrire en d\'etails au §4.

\end{xcomment}



Work will be divided into four main work packages: (1) procedural animation of isolated actors; (2) procedural anima-tion of interaction between actors; (3) authoring and real-time control; (4) user evaluations. Through the authoring tool (WP3), a script is elaborated by a theater director (WP4); it gives direction to group of actors which act out autono-mously the commands of the script to position toward each other and in the virtual space (WP2). The behaviors of each actor is computed taking into account their emotional states and social relations (WP1).

\endinput
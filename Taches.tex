
\subsection{Description des travaux par t�che / Description by task}
\begin{xcomment}  
Pour chaque tâche, d\'ecrire : 
les objectifs et \'eventuels indicateurs de succ\`es,
les personnes impliqu\'ees,
le programme d\'etaill\'e des travaux,
les livrables,
les contributions des personnes (le qui fait quoi ),
la description des m\'ethodes et des choix techniques et de la mani\`ere dont les solutions seront apport\'ees,
les risques et les solutions de repli envisag\'ees.
\end{xcomment}






\subsubsection{WP1 Kinesics}




\paragraph{Task11} Bidimensional models



\paragraph{Task12} Gesture and dialogue



\paragraph{Task13} Melange et extrapolation


\paragraph{Task11}  Choosing representations for full body motion ; representation learning ; transfer learning ; parameterisation of kinesic components. This can include neuro-muscular variables ; also the choice of kinematic trees rooted at the head ; also the grouping of kinematic variables into synergies ; etc. 

One classical distinction is between world-frame positions and joint angle variables In our case, we are making a strong statement that we will  study kinesic variables for all joints relative to the rigid body frame associated with the actor. This could be the ground floor position of the actor plus a rigid body position associated with the actor?s head.  Thus proxemic variables could be footsteps and head movements ; kinesic variables could be all other joint angles or joint positions in world coordinates.

\paragraph{Task12} Learning bi-dimensional models of actions and moods from a mocap dataset of six actions (walk, carry object, knock on door, throw object, lift object, move object) in 8 moods (neutral, happy, afraid, angry, anxious, sad, proud, shameful). Ideally, we would like to separate the action components from the mood components of motion and extrapolate the moods to other actions, and the actions to other moods. 

\paragraph{Task13} Learning models of gesture and facial expressions in dialogue situations.
Based on previous work on  visual prosody, we would like to learn joint models of gesture and  speech prosody. Ideally, this should be done without MOCAP data, using only audio and video processing, possibly enhanced with depth (kinect).

\paragraph{Task14}  Learning parameterized kinesic models. 

Bbecause our models are contextual, conditioned by the proxemic components, we should be able to change their velocity,  amplitude, direction and phase (in the way of parametric HMM models). This is challenging and needs to be investigated. 

\endinput

\subsubsection{WP2 Proxemics}



In this workpackage we are interested in modeling behaviors of group of agents while conversing and while moving around. We will pay particular attention at the social interaction of the agents during these activities. We will also develop an animation model that incorporate two models: statistical model as developed in WP1 and procedural model developed within the Greta plateform.


\paragraph{Task 2.1: Group behaviors during multi-way conversation}


In this task we will model multi-party conversation behaviors. We will focus on turn-taking management. While indication of what the agents would say to whom and when will be provided by a script (Task 3.1 and Task 4.X), the turn-taking model will instantiate which behaviors the agents will display. Gaze, body orientation, position in space are important cues for indicating who has the turn, who wants to keep it, to give it to someone, who listens? We will extend an existing turn-taking model (Ravenet et al., 2014) that is based on Sack?s model (Sack et al, 1974), that embeds F-Formation (Kendon, 1990) and that takes into account social attitude of the agents toward each other. This model is implemented as a state machine where the states are defined by the turn-taking and correspond to conversational roles. Transition between states is triggered when an agent changes conversational role. Attitudes vary the behavior of the agents such as their propensity to gaze at others. We will extend this model to simulate different configurations of speech overlap such as terminal overlaps, conditional access to the turn, and choral (Schegloff, 2000) as well as long silences when nobody takes the turn. We will add further states to encompass more conversational functions (eg greeting, word search?). We will also model that transitions from one state to another one can bring the agents of a group to be in the same state (parallel configuration as when greeting each other or laughing together).

A version of this model will be instantiated to model tri-partite interaction between two virtual actors and the audience (viewed as a virtual actor taking part of the interaction).


\paragraph{Task 2.2 : Group behaviors during stage movements}


This involves implementation of advanced � steering behaviors � such as follow, flee, separate, join, merge, enter stage, exit stage, etc.


This task will model agents? behavior when moving around in the environment. The animation of the virtual agent doing some tasks will be given by WP1. It will not focus on path planning as this information will be provided by a script (Task 3.1 and Task 4.X). Rather it will model how agents perform displacement in social settings. Gaze direction, body orientation and spatial distance to other agents will be computing for different ?steering behaviors?. These features will be modeled through different synchronization mechanisms: moving in synch, moving ahead, following, etc. They evolve dynamically in function of each agent?s position and orientation in space. The basic animation of the agent, ie without any influence from surrounding agent, is given by WP1. To simulate the dynamic evolution of agents' behaviors we will make use of Neural Network simulation (Prepin 2013) where we can render how behaviors of one actor can act on behaviors of other actors (eg walking powerfully toward an actor with an angry expression will result in moving backward of another actor with a less dominant attitude. Mutual coupling of behaviors will be modeled as emerging from such action-reactive behavior simulation (Prepin 2013) ensuring not only the synchronization between actors? behaviors but also their mutual influence.

As for Task 2.2, a version of this model will be instantiated to consider the audience as one virtual actor.



\paragraph{Task 2.3:  Combination of statistical and procedural models.}


In this task we will develop an animation model that will merge animations coming from statistical model developed in WP1 and procedural model developed in WP2 (Task 2.1 and Task 2.2). This blend is required for the interaction settings where behaviors of the agents are driven by both animation models.

The procedural model relies on forward and inverse kinematic models (Huang, 2012). It controls the arms position, gaze direction and body orientation. The statistical model (from WP1) controls the whole body. Our animation blender model will work at the modalities level and will also incorporate movement propagation; that is how motion of one body part affects other body parts. At first, the animation blender model will merge whole body motion computed by the statistical model as specific body motion computed by the procedural model. More precisely, arms position, gaze direction and body orientation outputted by the procedural model will be viewed as constraints to be reached. These motions will be added onto the animation computed by statistical model; the position of the arms, head and torso computed by the procedural model will overwrite those computed by the statistical model. In a second step, the animation blender model will incorporate propagation of movements. To compute movement propagation we will develop a statistical model that learns which motion is due to action and which motion is due to movement propagation.




\endinput


\subsubsection{WP3 Authoring}

\paragraph{Task31} Specification of a dramatic language of verbs (actions, speech acts, movements) and adverbs (moods, attitudes, dramatic effects) for directing actors ; define cues as synchronisation points between actors ; define parallel and sequential behaviors ; etc.

Part of this language will be devoted to stage blocking / movement

Part of this language wil be devoted to dialogue

\paragraph{Task32}  Compilation of the language into a finite state machine and/or Petri net ; allowing real-time execution of the dramatic score ; user interface for directing actors by sketching stage floor plans and composing the dramatic score ; one line per actor per motion component (proxemic behaviors, kinesic actions, kinesic moods, speech acts, etc.)

\paragraph{Task33} Real-time execution of the dramatic score ; real-time combination of proxemic (procedural) and kinesic components of motion ; non-deterministic motion generation ; synchronization to cues ; real-time skinning and advanced 3D animation ; integration of physically-based secondary animation (skin, hair, clothes, etc.)

This includes integration of the GRETA BML realizer with IMAGINE animation ; and real-time integration of the statistical models of motion with the procedural animation components.

\endinput

\subsubsection{WP4 User evaluation and validation}

\paragraph{Task41} Scenarios.

Writing scenes with didascalia 

Dialogue scenes with groups of 2 or  3  actors using a choice of didascalia

Movements with groups of 2 or 3 actors using a choice of didascalia

Alternations of dialogue and stage movements in theatre scenes with 2 or 3 actors

A possible choice would be "the augmentation", a play by Georges Perec with a large number of variations
on a single theme (an employee asks an augmentation from his boss in the presence of his secretary).



\paragraph{Task42} Validation of the interaction.

Is the dramatic language adequate ? useful ?  efficient  ? 

Is the dramatic score interface  adequate ? useful ?  efficient  ? 

Is the stage floor plan sketching tool adequate ? useful ?  efficient  ? 



\paragraph{Task43} Validation of the animation

Dialogue scenes  with groups of 2, 3 and 4 actors.

Silent stage movements of groups of 2, 3 and 4 actors, as in opera synched to music

Combination of dialogue and action for scenes with 2 actors



\endinput




%Additional notes
%
%
%We will dedicate joint research between Inria and LIF to make it easy to extend our database of actions and attitudes
%using video, rather than motion capture. This will necessitate fundamental research in transfer learning (so that the sparse
%data obtained from video can benefit from the dense data obtained with motion capture) and video processing.  Following the
%methodology of  gesture controllers \cite{Levine2010}, where the gesture are controlled directly by speech prosody features
%extracted from real actors voices, it appears possible to drive expressive and plausible gestures and body movements from
%visual signatures of actions and attitudes extracted from example videos. 
%
%We will use  our previous work in actor and action recognition \cite{Weinland06,Weinland07,Gandhi13} to detect and recognize
%actors and their actions in real movies ; and extract visual signatures of the corresponding actions and attitudes. Based on this
%analysis, we will learn joint statistical models for driving gesture controllers from those video signals.







\subsection{Calendrier des t�ches, livrables et jalons / Tasks schedule, deliverables and milestones}
\begin{xcomment} 
Pr\'esenter sous forme graphique un \'ech\'eancier des diff\'erentes tâches et leurs d\'ependances (diagramme de Gantt par exemple).
Pr\'esenter un tableau synth\'etique de l'ensemble des livrables du projet (num\'ero de tâche, date, intitul\'e, responsable).
Pr\'eciser de façon synth\'etique les jalons scientifiques et/ou techniques, les principaux points de rendez-vous, les points bloquants ou al\'eas qui risquent de remettre en cause l'aboutissement du projet ainsi que les r\'eunions de projet pr\'evues.
\end{xcomment}


\begin{figure}[htbp]
\begin{center}
\includegraphics[width=\linewidth]{ganttchart_dada.png}
\caption{GANTT diagram of tasks and ressources for DADA.}
\label{default}
\end{center}
\end{figure}


\subsubsection{Milestones}
 
The project will be organized in four main phases, separated by three milestones at $T_0+12$, $T_0+24$ and $T_0+36$:

\vspace{1cm}
\begin{center}
\begin{tabular}{|c|c|}
\hline
Phase 1 ($T_0 \rightarrow T_0+12$) & Specifications and scenarios. \\\hline
Milestone 1 ($T_0+12$) & Delivery of  specifications and scenarios.\\\hline
Phase 2 ($T_0+12 \rightarrow T_0+24$) & Research and development for first prototype.\\\hline
Milestone 2 ($T_0+24$) & Delivery of  first prototype.\\\hline
Phase 3 ($T_0+24 \rightarrow T_0+36$) & Research and development for second protoype.\\\hline
Milestone 3 ($T_0+36$) & Delivery of  second prototype.\\\hline
Phase 4 ($T_0+36  \rightarrow T_0+42$ & Final user evaluations.\\\hline
\end{tabular}
\end{center}


\subsubsection{Deliverables}
The project includes 12 deliverables which will be delivered at the three milestones. 

%\vspace{1cm}
\begin{center}
\begin{tabular}{|c|p{10cm}|c|c|}
\hline
Number & Title & Date & Contributor\\\hline
L1.1  & Report on the state of the art for statistical models for animation synthesis &  $T_0+12$ & LIF\\\hline
L1.2  & First version of the models : Prototype (software) and its documentation (Report on the models developed) & $T_0+24 $  & LIF\\\hline
L1.3  & Second version of the models : Prototype (software) and its documentation (Report on the models developed)  & $T_0+36$ & LIF \\\hline
L2.1  & Report on the state of the art of proxemics models in computer animation&   $T_0+12$ & LITC \\\hline
L2.2  &  First version of the models : Prototype (software) and its documentation (Report on the models developed) & $T_0+24$ &LITC \\\hline
L2.3  &  Second version of the models : Prototype (software) and its documentation (Report on the models developed) &  $T_0+36$ & LITC \\\hline
L3.1  & Specification of the dramatic score language &  $T_0+12$  & Inria \\\hline
L3.2  & First version of authoring tools and execution environment : Prototype (software) and its documentation &  $T_0+24$  & Inria \\\hline
L3.3  & Second version of authoring tools and execution environment : Prototype (software) and its documentation &  $T_0+36$   & Inria \\\hline
L4.1  & Selection of example scenes and examples: collection of  one actor directing exercices   & $T0+12$ & Paris 8\\\hline
L4.2  & Preliminary revaluation report: Animated scenes with one actor, on the exercises defined in task 4.1. & $T0+30$ & Paris 8 \\\hline
L4.3  & Final evaluation report: Animated scenes with three virtual actors, showing exercises and selected examples from "L'augmentation" by Perec. & $T0+42$ & Paris 8 \\\hline
\end{tabular}
\end{center}



 
\endinput
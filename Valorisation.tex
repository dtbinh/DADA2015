


\section{Strat\'egie de valorisation, de protection et d'exploitation des r\'esultats / Dissemination and exploitation of results. intellectual property}
\begin{xcomment}   
A titre indicatif : 2 pages maximum pour ce chapitre.
Pr\'esenter les strat\'egies de valorisation des r\'esultats :
la communication scientifique,
la communication aupr\`es d'autres communaut\'es scientifiques et du grand public, notamment la promotion faite à la culture scientifique et technique. Si un budget sp\'ecifique est pr\'evu à cet effet, le sp\'ecifier et l'identifier dans une tâche de la proposition (voir § 3.1).
les r\'esultats attendus en mati\`ere de valorisation,
les retomb\'ees scientifiques, techniques, industrielles, \'economiques, '
la place du projet dans la strat\'egie industrielle des entreprises partenaires du projet,
les autres retomb\'ees (normalisation, information des pouvoirs publics, formation dans l'enseignement sup\'erieur, ...),
les \'ech\'eances et la nature des retomb\'ees technico-\'economiques attendues,
l'incidence \'eventuelle sur l'emploi, la cr\'eation d'activit\'es nouvelles, '

Pr\'esenter les grandes lignes des modes de protection et d'exploitation des r\'esultats.
\end{xcomment}


\endinput





\section{Strat\'egie de valorisation, de protection et d'exploitation des r\'esultats / Dissemination and exploitation of results. intellectual property}
\begin{xcomment}   
A titre indicatif : 2 pages maximum pour ce chapitre.
Pr\'esenter les strat\'egies de valorisation des r\'esultats :
la communication scientifique,
la communication aupr\`es d'autres communaut\'es scientifiques et du grand public, notamment la promotion faite à la culture scientifique et technique. Si un budget sp\'ecifique est pr\'evu à cet effet, le sp\'ecifier et l'identifier dans une tâche de la proposition (voir § 3.1).
les r\'esultats attendus en mati\`ere de valorisation,
les retomb\'ees scientifiques, techniques, industrielles, \'economiques, '
la place du projet dans la strat\'egie industrielle des entreprises partenaires du projet,
les autres retomb\'ees (normalisation, information des pouvoirs publics, formation dans l'enseignement sup\'erieur, ...),
les \'ech\'eances et la nature des retomb\'ees technico-\'economiques attendues,
l'incidence \'eventuelle sur l'emploi, la cr\'eation d'activit\'es nouvelles, '

Pr\'esenter les grandes lignes des modes de protection et d'exploitation des r\'esultats.
\end{xcomment}


A consortium agreement will be signed in the course of the first year to define a common 
intellectual property rights policy. In order to patent their inventions, partners should seek authorization
from all other partners.  The other partners cannot prevent the patent unless they have prior art. 
The partners  can ask to be mentioned as co-inventors if they can demonstrate that they contriuted to the invention.

The consortium is composed exclusively of academic partners, whose goal is primarily  to disseminate and publish their research work, not to commercialize software.
We will seek to publish our research results in top ranked conferences and journals, including SIGGRAPH, Eurographics, NIPS, ICML, IJCAI, AAAI, IVA, AAMAS, etc. In addition to scientific papers, the consortium will also seek to publish paper describing  artistic applications of the DADA framework, possibly as demos, art papers and exhibits at SIGGRAPH, Ars Electronica, Laval Virtual and other similar venues. 

On the other hand, some of the inventions may have a commercial value. We will seek advice from
a small board of industry experts  from relevant  French companies (Goalem, Quantum Dream, Ubisoft, Dassault Syst�mes) 
to detect and address such cases  and make sure that potentially important inventions are protected and made available to them.


\endinput


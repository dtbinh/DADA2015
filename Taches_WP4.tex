\subsubsection{WP4 User evaluation and validation}


\begin{center}
\begin{tabular}{|l|l|}\hline
WP4 &  User evaluation and validation \\\hline
Responsable &  Paris 8  \\\hline
Participants &  ECM, Inria, LTCI\\\hline
Duration  &   42 months \\\hline
Objectives &  Deployment, testing and evaluation of DADA prototypes  \\\hline
Content &  \\\hline
Task 4.1 & Exercises and scenarios   ($T_0 \rightarrow  T_0+12$) \\\hline
Task 4.2 & User evaluation of interaction with a single virtual actor  ($T_0 \rightarrow  T_0+30$) \\\hline
Task 4.3 &  User evaluation of interaction with multiple virtual actors  ($T_0 \rightarrow  T_0+42$) \\\hline
\end{tabular}
\end{center}

\paragraph{Task 4.1 : Exercices and rehearsal scenarios}

Based on the theatrical methodology exposed in task 3.1, we'll design short exercices inspired by the play entittled  "L'augmentation" by Georges Perec. Writen in the Oulipo style, the play declines a large number of variation developing the recurrent theme of an employee asking an augmentation from his boss, with the assistance of his secretary.  Basic  actions are used in multiple ways to express subtle strategies for convincing the boss. We will find the proper vocabulary to describe and parameterize those basic actions, taken as exercices that could be done by a real actor, and that will be perform by a virtual agent under the direction of the real stage director. 

The catalogue of basic actions will be realized at first with one actor to test the efficiency of the authoring tool. The evaluation will be done by digital artist team (INREV-ATI) and theater team (Sc\`enes du monde, cr\'eation et savoir critiques) on two levels. On the one hand, we will evaluate the ergonomy and usability of the tools to reach proper results, and the quality of the virtual rendering, both on visual aspect and narrative coherence.  On the other hand, we will precisely measure the ambitus of espressivness we could ask to the virtual actor, and the possibilities of combining and finding emergent solutions to artistic intuitions in a creative way. 

%How the tool empowers the creativity and the imagination of the director and how to limit the risks of frustration by non 

We will start with simple exercises, as soon as the first deliverables of WP3 will be ready, in order to orient properly the complexity of the score, the writing tool and the virtual behaviors. We will increase step by step the difficulty of the exercices in both movement ability and emotionnal espressivity. The exercices will also be conceived as a way to learn how to use the tool for directing virtual actors. We will also control the necessity to keep basic rules of behavior when complexity of exercices increase, i. e. the stability of the directing process. This is a specific quality of working with a troup of actors : a common gestual and emotionnal vocabulary shared by actors, directors and audience helps to bring strong creative proposals.  

Following the development of the authoring tool, we will develop new exercises focusing on interactions between two and three actors, in the continuity of the collection produced in 4.1 and the specifications developed in WP1 and WP2, and more specifically in 3.2. 


\paragraph{Task 4.2 : User evaluation of interaction with a single virtual actor}

Based on the first prototype, and the exercises for one actor, we will combine artistic digital approach with theatrical knowledge, in order to conduct the development of the authoring tool in a proper direction. We will give precise evaluation of the needs in the rendering and in the ergonomy to respect a creative process and to reach espressive result. For instance, are the dramatic language, the performance score, the stage floor plan sketching tool adequate, useful, efficient ? Short extracts of L'augmentation will be chosen by Georges Gagner\'e and virtually staged with the authoring tools with the help of a postdoc researcher for reporting the bugs, precising the directing needs both ergonomically and creatively in connexion with WP3,  and documenting the authoring tool for novice   users.  Those first results about very short performances will be submitted to a wide range of researchers and professionals of different artistic and cultural fields to evaluate the quality of the theatrical interactions both between virtual actors, and between virtual actors and real spectators (animation, video game and theater)  


\paragraph{Task 4.3 : User evaluation of interaction with multiple virtual actors}

A postdoc researcher, specialized in actor directing, will learn "How to use the authoring tool ?" and complete the documentation process for students and professional artists. Together with Georges Gagner\'e, he will write the performance score for entire scenes of "L'augmentation". Two different users points of view on the authoring tool will be confronted, and will help to design the proper strategy for a broader dissemination. The Inrev-ATI and theater teams will also organize meetings to introduce the tool to multiple artistic communities, in the very creative context of the Laboratory of Excellence in Arts and Human Mediations ( Labex ARTS H2H) They will organise workshops to learn  and practice the authoring tool especially for his own creative purpose. The postdoc researcher will assist professional stage directors to realize virtual direction of short extracts from other pieces of the theatrical repertoire and confront the tool to different styles of actor direction. Explicitation interviews will be conducted with  users of the DADA prototype to analyze the technological constraints that could limit the creative approach. 

Using the collections of exercices produced in 4.1 and 4.2, a specific approach of the authoring tool  around pedagogical issues will also be dedicated to master students of the IDEFI CREATIC innovative training program, including Paris 8 University,  West Paris Nanterre-La D\'efense University, the Maison des Sciences Humaines Paris Nord (North Paris Centre for Human Sciences), the Conservatoire National Sup\'erieur d'Art Dramatique (National Drama Academy), the National Archives and 37 foreign partners. 

Finally, we will also evaluate the quality of the animation produced with the authoring tool with a variety of audiences.  


\begin{tabular}{|l|p{10cm}|l|}\hline
Deliverables & Name and content  & Date  \\\hline
L4.1  & Selection of example scenes and examples: collection of  one actor directing exercices allowing an ergonomic appropriation of the authoring tool and its animation possibilities, in a theatrical creativity context.   & $T0+12$ \\\hline
L4.2  & Preliminary revaluation report: Animated scenes with one actor, on the exercises defined in task 4.1. Results of user study on the subjective quality of the authoring tools and the produced animation.  Collection of  three-actor exercises allowing an ergonomic appropriation of the authoring tool and its animation possibilities in a theatrical creativity context. Documentation of  the authoring tool for theatre directors.  & $T0+30$ \\\hline
L4.3  & Final evaluation report: Animated scenes with three virtual actors, showing exercises and selected examples from "L'augmentation" by Perec. User documentation of the authoring tools from a pedagogical context perspective. Results of user studies on subjective evaluations of the tools and produced animations in the  professional and the academic fields. & $T0+42$ \\\hline
\end{tabular}

\endinput


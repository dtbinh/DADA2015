\subsubsection{WP4 User evaluation and validation}


\begin{center}
\begin{tabular}{|l|l|}\hline
WP4 &  User evaluation and alidation \\\hline
Responsable &  Paris 8  \\\hline
Participants &  ECM, Inria, LTCI\\\hline
Duration  &   \\\hline
Objectives &   \\\hline
Content &  \\\hline
Task 41 & Scenarios  \\\hline
Task 42 &  Validation of the animation \\\hline
Task 43 &  Validation of the interaction  \\\hline
\end{tabular}
\end{center}

\paragraph{Task41: Scenarios.}

Based on the theatrical methodology exposed in task 3.1, we will create shortl exercices inspired by the theater play "L'augmentation" by  Georges Perec. Writen in the Oulipo style, the play declines a large number of variation developping the recurrent theme of an employee asking a raise (augmentation) from his boss, with the assistance of his secretary. Basic  actions are used in multiple ways to express subtle strategies for convincing the boss. We will find the proper vocabulary to realise this basic actions, taken as exercices that could be done by a real actor, and that will be perform by a virtual agent under the direction of the real stage director.

The catalogue of basic actions will be realized at first with one actor to test the efficiency of the authoring tool. The evaluation will be done by digital artists, researchers and students in theatre studies 
at Paris 8 on two levels : on one hand, evaluation of the tool's ergonomy to reach proper results, and of the quality of the virtual rendering, both on visual aspect and narrative coherence. On the other hand, we will precisely measure the ambitus of espressivness we could ask to the virtual actor, and the possibilities of combining and finding emergent solutions to artistic intuitions in a creative way. How the tool empowers the creativity and the imagination of the director and how to limiting risks of frustration ?

We will start with simple exercises, as soon as the first deliverables of WP3 will be ready, in order to orient properly the complexity of the score, the writing tool and the virtual behaviors. We will increase step by step the difficulty of the exercices in both movement ability and emotionnal espressivity. The exercices will also be designed as a way to learn how to use the tool for directing virtual actors. We will also control the necessity to keep basic rules of behavior when complexity of exercices increase, i. e. the stability of the directing process. This is a specific quality of working with a troup of actors : a common gestual and emotionnal vocabulary shared by actors, directors and audience helps to bring strong creative proposals.



\paragraph{Task42: User evaluation and validation of the animation.}

Short extracts of L'augmentation will be chosen by Georges Gagner� and virtually staged with the authoring tools with the help of a postdoc researcher for reporting the bugs, precising the directing needs both ergonomically and creatively in connexion with WP3,  and documenting the authoring tool for new other users.

The quality of the animation will be evaluated subjectively and guidelines  for future work will be included in the deliverable report.


\paragraph{Task43: User evaluation and validation of the interaction.}

Following the development of the authoring tool, we will develop new exercises focusing on interactions between two and three actors, in the continuity of the collection produced in 4.1 and the specifications developped in 3.2. Using the same methodology, we will combine artistic digital approach with theatrical knowledge, in order to conduct the development of the authoring tool in a proper direction. We will give precise evaluation of the needs in the rendering and in the ergonomy to respect a creative process and to reach espressive result. For instance, are the dramatic language, the dramatic score, the stage floor plan sketching tool adequate, useful, efficient ? 

This first results about very short performances will be submit to a wide range of researchers and and professionnals of different artistical and cultural fields to evaluate the quality of the theatrical interactions both between virtual actors, and between virtual actors and real spectators (animation, video game and theater)

A postdoc researcher, specialized in actor directing, will learn to use the authoring tool  and complete the documentation process for students and professional artists end users. He will also write entire scenes of "L'augmentation", together with Georges Gagner�. Two different users points of view on the authoring tool will be confronted, and will help to design the proper strategy for a broader dissemination. We will organize internal demos and meetings to introduce the tool to multiple artistic communities, in the very creative context of the Laboratory of Excellence in Arts and Human Mediations ( Labex ARTS H2H)

%They will organize workshops to learn  and practice the authoring tool especially for his own's creative puprose. The postdoc researcher will assist professional stage directors to realize virtual direction of short extracts from other pieces of the theatrical repertoire and confront the tool to different styles of actor direction. 

Interviews will be conducted with users to analyse the technological constraints that could limit the creative approach. Using the collections of exercices produced in 4.1 and 4.2, a specitif approach of the authoring tool  around pedagogical issues will also be dedicated to master students of the IDEFI CREATIC innovative training program, including Paris 8 University,  West Paris Nanterre-La D�fense University, the Maison des Sciences Humaines Paris Nord (North Paris Centre for Human Sciences), the Conservatoire National Sup�rieur d'Art Dramatique (National Drama Academy), the National Archives and 37 foreign partners. We will also evaluate the quality of the authoring tool rendering dimension  with a large public.of spectators.

\paragraph{Deliverable L4.1: Scenarios and exercises.} 

Collection of  exercises for directing one actor, allowing an ergonomic appropriation of the authoring tool and its animation possibilities, in a theatrical creativity context.

Collection of exercises for  directing three actors, allowing an ergonomic appropriation of the authoring tool and its animation possibilities in a theatrical creativity context.

Collection of scenes from the play "L'augmentation" suitable for evaluation and validation.


\paragraph{Deliverable L4.2: Preliminary evaluation report.}

This will include an evaluation of  the animations produced in WP1, WP2 and WP3 during the first part of the project, 
based on the selected single actor's exercises,  as well as an initial evaluation of the usability of the authoring tools, 
again using the single actor's exercises.

The report will include a documentation for the tools, written specifically for directors and artists.

The report will include guidelines for the second prototype.

\paragraph{Deliverable L4.3: Final evaluation report.}

This will include an evaluation of  the animations produced n WP1, WP2 and WP3 during the second part of the project, 
based on the selected three-actor exercises and the selected  scenes from "L'augmentation"; and an evaluation of the usability
of the tools proposed in the second prototype, agin using the three-actor exercises and selected scenes from "L'augmentation".

The final report will include a documentation for the tools, written specifically for directors and artists.

The final report will include the results of surveys and user studies on the reception of the tools and the resulting animations 
by the professional and academic fields.

The final report will include guidelines for future research.


\begin{tabular}{|l|l|l|}\hline
Deliverables & Name and content  & Date  \\\hline
L4.1  & Selection of example scenes and examples & \\\hline
L4.2  & Preliminary revaluation report. & \\\hline
L4.3  & Final evaluation report & \\\hline
\end{tabular}

\endinput

